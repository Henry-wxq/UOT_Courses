\documentclass[12pt]{article}

\usepackage{enumerate}
\usepackage{graphicx}
\usepackage{geometry}
\usepackage{float}
\usepackage{appendix}
\usepackage{color}
\usepackage{amsmath}
\usepackage{pifont}
\usepackage{fancyhdr}
\usepackage{enumitem}
\usepackage{amssymb}
\fancyhf{}


\geometry{a4paper}
\geometry{left = 2cm}
\geometry{right = 2cm}
\geometry{top = 3cm}
\geometry{bottom = 3cm}

\pagestyle{fancy}
\lhead{STA130H1}
\rhead{Capstone Project Proposal}
\cfoot{\thepage}

\title{STA130H1 Capstone Project Proposal}

\author{Xuanqi Wei, Shujun Yang, Riyad Valiyev, Nicolas Dias Martins}

\date{\today}

\begin{document}


\maketitle
\thispagestyle{empty}

\newpage

\tableofcontents
\thispagestyle{empty}

\newpage

\setcounter{page}{1}

\section{Introduction}

The following are sections of `Capstone Project Proposal' for STA130H1 in 2023 winter session written by Xuanqi Wei, Shujun Yang, Riyad Valiyev and Nicolas Dias Martins.

\

\noindent Firstly, we will introduce the general data we will use in the Capstone Project. We are analyzing data from the data set `Galaxies: Galaxy Zoo' in our project. This data set consists of the Galaxy Zoo data (data taken from the Galaxy Zoo volunteers' classifications of 209,293 galaxies) and the associated tables from the NASA-Sloan Atlas (alternative properties of 641,409 galaxies derived through various algorithms), both provided by Mike Walmsley. The data of the tables can be matched based on the ID (`iauname') of the galaxies.

\

\noindent Secondly, we will introduce the structure of our `Capstone Project Proposal'. As listed in the Content section, we will provide detailed description and explanation regarding 3 research questions, contribution and timeline in sequence. To proceed with, in each research question, we will clearly state the data it analyses, the method it conducts and the visualization it illustrates as well as their appropriation and justification. 

\

\noindent Finally, it is a pleasure to sincerely acknowledge our indebtedness to our TA, whom we warmly thank for his observation and suggestions, which materially improved the proposal draft. 

\newpage
\section{Research Questions}

\subsection{RQ1: How is the galaxy's total size related to the percentage of light within the radius?}

To answer the research question 1, we will firstly determine the data which will be used in analysis including its relation to the research question 1. Following by describing the method that will be apply on this question with justifying the relationship to research question 1 and ending with the description of visualization which includes relation to the research question, plots of two graphs and justification about the appropriation. 


\subsubsection{Data}

The data we will use in research question 1 comes from `Galaxy Zoo Tabular Data Contents', more specifically, \textcolor{cyan}{petro\_theta}, which is an estimate of a galaxy's total size in radius, \textcolor{cyan}{petro\_th50}, which is an estimate of the galaxy's size in radius where 50\% of the light is inside the radius and 50\% of the light is outside the radius, and \textcolor{cyan}{petro\_th90}, which is an estimate of the galaxy's size in radius where 90\% of the light is inside the radius and 10\% of the light is outside the radius. According to the aforementioned descriptions, the 'galaxy's total size' is related to `petro\_theta', while the percentage of light within the radius' is related to both `petro\_th50' and `petro\_th90'. 

\subsubsection{Method}

The method we will use in research question 1 is \textcolor{cyan}{bootstrapping}. The method of bootstrapping allows us to, by getting a random sample from the data set, use random sampling with replacement to simulate a sampling distribution of a specific sample statistics. In research question 1, we will use bootstrapping to simulate sampling distributions of the mean of three sample statistics (by using the same random sample): `petro\_theta', `petro\_th50', and `petro\_th90'. Since we want to discover `how the galaxy's total size is related to the percentage of light within the radius' in research question 1, we will compare the three distribution graphs from the bootstrapping process in order to understand the relationship between the different sample statistics. 

\subsubsection{Visualization}

The visualizations for research question 1 are three different \textcolor{cyan}{sampling distribution graphs} (which is a plot that shows the distribution of a statistics sample):

\begin{enumerate}
	\item The first sampling distribution graph is related to the mean values of the `petro\_tetha' statistic from the boot samples. We will set the x-axis to be the range of possible mean values from the boot samples, while we will set the y-axis to be the frequency (count) of a specific mean value. 
	\item The second sampling distribution graph is related to the mean values of the `petro\_th50' statistic from the boot samples. We will set the x-axis to be the range of possible mean values from the boot samples, while we will set the y-axis to be the frequency (count) of a specific mean value. 
	\item The third sampling distribution graph is related to the mean values of the `petro\_th90' statistic from the boot samples. We will set the x-axis to be the range of possible mean values from the boot samples, while we will set the y-axis to be the frequency (count) of a specific mean value. 
\end{enumerate}

\noindent The graphs are appropriate for the intended audience as sampling histograms were taught during lectures and are an easy way of visualizing the data. In addition to that, since we want to discover `how the galaxy's total size is related to the percentage of light within the radius' in research question 1, we will compare these three distribution graphs in order to understand the relationship between the different sample statistics we are analyzing (`petro\_theta', `petro\_th50', and `petro\_th90').


\newpage

\subsection{RQ2: Do the galaxies farthest to the Earth differ significantly from the closest ones in terms of their total luminosity?}

To answer the research question 2, we will firstly determine the data which will be used in analysis including its relation to the research question 2. Following by describing the method that will be apply on this question with justifying the relationship to research question 2 and ending with the description of visualization which includes relation to the research question, plots of two graphs and justification about the appropriation. 

\

\noindent $H_0$ = The mean of the total luminosity of the closest galaxies is not significantly different than that of the farthest ones.

\

\noindent $H_A$ = The mean of the total luminosity of the closest galaxies is significantly different than that of the farthest ones.

\subsubsection{Data}

The data we will be using in research question 2 is from the `Galaxy Zoo Tabular Data Contents' with the data set `nsa\_v1\_0\_1\_key\_cols.parquet', provided by Mike Walmsley. The variables involved in investigating the given question are \textcolor{cyan}{elpetro\_absmag\_r}, which is the estimate of the total luminosity or, in other words, intrinsic brightness of the galaxies and \textcolor{cyan}{redshift} - a measure displaying how far away the galaxy is from us. Every value under the variable `elpetro\_absmag\_r' is expressed in terms of absolute magnitudes. 

\

\noindent Since our question aims to address two main attributes of the galaxies, total luminosity and distance from the Earth, the aforementioned variables are relevant. To be  precise, we will make use of two different samples from the population of all galaxies: a sample of the farthest 50 galaxies (50 galaxies with the highest redshift values) and the closest 50 galaxies (50 galaxies with the lowest redshift values).

\subsubsection{Method}

Our method for the second research question will be \textcolor{cyan}{test hypothesis}. As expressed above, our null hypothesis is `the difference of the means of the total luminosity values of the galaxies in the first and second samples being equal (or close to equal)' while the alternative hypothesis is that there is a noticeable difference between the means. We will first look into the question of whether there is indeed a difference between the means of the total luminosities between the two samples, or not, and get our measured test statistic. Following this, we'll use our null hypothesis (which will be a `close to 0' value less than some certain small enough number) to determine the probability options. Our test statistic will be the difference between the two means, which we will use to simulate the sampling distribution of that test statistic under the null hypothesis, e.g. 2000 times, with replacement. Then, we'll graph the plot of the distribution based on our simulation results and calculate the 2-sided p-value according to the measured test statistic from earlier. Based on the rejection rule, we'll either reject or fail to reject our null hypothesis depending on whether or not the p-value falls under the pre-specified alpha threshold (significance level). 

\subsubsection{Visualization}

The visualizations for this research questions will be a \textcolor{cyan}{bar graph} and \textcolor{cyan}{histogram}. 

\

\noindent The former will be used to compare the actual means of the total luminosity values between the given samples, with its x-axis representing sample names to answer the question of `which bar belongs to which sample' and its y-axis showing the numerical mean value. Since there will be two overall samples, the plot will consist of two charts each mapping to a specific value of mean on y-axis. In this scenario, this plot type is the best choice, as the x-axis will contain discrete values rather than continuous. 

\

\noindent Apart from this, we'll also use histogram to visualize the distribution of the mean values we get from the sampling simulations under the null hypothesis. The choice of a histogram as our plot type for this part is appropriate since the x-axis will denote the possible mean values that will be produced as a result of the sampling simulations, which is a continuous numerical data, and y-value will represent their frequencies, i.e. the number of times the mean values in each interval appear upon the simulations. Histogram is also very comfortable in the sense that we will be able to adjust the bin numbers of the graph (narrow down the intervals) depending on how detailed we want to convey the information to the reader.

\newpage

\subsection{RQ3: How well can a simple linear regression model predict the galaxy's apparent brightness from redshift?}

To answer the research question 3, we will firstly determine the data which will be used in analysis including its relation to the research question 3. Following by describing the method that will be apply on this question with justifying the relationship to research question 3 and ending with the description of visualization which includes relation to the research question, plots of two graphs and justification about the appropriation. 


\subsubsection{Data}

The data we used in research question 3 comes from `Galaxy Zoo Tabular Data Contents', more specifically, they're \textcolor{cyan}{redshift}, which is related to how far away that galaxy is from us, and, \textcolor{cyan}{sersic\_nmygy\_r}, which is an estimation used to estimate the galaxy's apparent brightness (which depends on how far away it is) in units of magnitudes. 

\

\noindent According to the description of redshift and sersic\_nmygy\_r above, `galaxy's apparent brightness' in the research question 3 is revealed using the sersic\_nmygy\_r data and the `redshift' in the research question 3 is revealed using the redshift data.

\subsubsection{Method}

The method we used in research question 3 is \textcolor{cyan}{simple linear regression}. To describe the method, simple linear regression is a statistical method that allows us to summarize and study relationships between two continuous or quantitative variables. One variable, denoted x, is regarded as the predictor variable; The other variable, denoted y, is regarded as the response variable. For simple linear regression, it concerns the study of only one predictor variable. To determine how well it relates, we'll estimate the regression line and then use the t-test to determine if the slope, $\beta_1$, of the population regression line is 0.

\

\noindent In relation to research question 3, since we want to know `how well a simple linear regression model can predict the galaxy's apparent brightness from redshift', we set the `redshift' as the predictor variable and `the galaxy's apparent brightness' as the response variable, using simple linear regression method described above to estimate the regression line and apply t-test to evaluate the wellness of relationship. 

\subsubsection{Visualization}

The visualizations for research question 3 are the \textcolor{cyan}{scatterplot} and the \textcolor{cyan}{fitted line plot}.

\

\noindent For the \textcolor{cyan}{scatterplot}, it is generally used when the values of the variables of the y-axis are thought to be dependent upon the values of the variable of the x-axis, which is exactly the case in research question 3 as we set `redshift' as the predictor variable and `the galaxy's apparent brightness' as the response variable. Therefore, in the scatterplot, we'll set the x-axis as `redshift' and the y-axis as `the galaxy's apparent brightness'. To justify the appropriation, since we expect two variables to be dependent, it successfully satisfied the circumstances when using the scatterplot, which means it's appropriate.

\noindent For the \textcolor{cyan}{fitted line plot}, it is generally used when obtaining the estimated regression function between a response y and a predictor x, which is as well exactly matches the case in research question 3 as we set `redshift' as the predictor variable and `the galaxy's apparent brightness' as the response variable. Therefore, in the fitted line plot, we'll set the x-axis as `redshift' and the y-axis as `the galaxy's apparent brightness', which is the same set as in the scatterplot. To justify the appropriation, although we already used the scatterplot graph to illustrate the dependent variables, the fitted line plot besides contains the shows a regression line superimposed on the data, providing much more accuracy when solving the research question 3, meaning it's appropriate. 


\newpage

\section{Contribution}
After team discussion, we regard the research question 3 the hardest research question. To ensure that the contributions of each group member is roughly an equal share to the overall workload, besides assigning a part of the whole procedure, we also assign each person a main research question to focus on. 

\begin{enumerate}
	\item Xuanqi Wei (RQ3): Conceptualization, Methods Determination, Coding;
	\item Shujun Yang (RQ3): Data Cleaning, Introduction Section;
	\item Riyad Valiyev (RQ2): Visualization, Conclusions Summarizing;
	\item Nicolas Dias Martins (RQ1): Result Summarizing, Reviewing and Editing.
\end{enumerate}

\section{Timeline}
\begin{enumerate}
	\item By end of Feb: Submit the proposal for project. Each group member will choose their own questions to work on and share with other group members about their research questions as well as their dataset, methods, and visualizations they will use. The research questions will  be accepted only when every group member understands all the concepts of the research questions, and agree it is a meaningful question.
	\item By mid-Mar: Each member should try to finish the main part (coding, analysis) of their own research questions. A small meeting and frequent conversation will help with all the problems we faced during the process, the small meeting will help us to share our experiences and tell the difference between expectation and reality and how we deal with that so that we can make sure everyone is on the same page and all the adjustments we make are reasonable.
	\item By Mar 30: All the members should finish their work and come up with a conclusion, each of us should send the related slides to one of our members so we can get a completed power point. The whole report should be done and check by all the group members even by TA. Two meetings will happened during this time, the first one is for the connection of reports and slides, the second one is the practice for presentation.
	\item By Apr 5: All the work should be done and ready to submit. All the problems should be fixed by conversation in group and the help form TA. 



\end{enumerate}


\end{document}
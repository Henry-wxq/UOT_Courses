% CSC165 Winter 2023: Problem Set 0
% Template to use to complete Problem Set 0.
% Note that using this template is *optional*: it provides a nice foundation
% for getting started with LaTeX, but you aren't required to use it!
% If you are using Overleaf, you'll want to upload this file to your account.
% Before modifying this file, we recommend trying to compile it as-is
% to see what the basic template gives.

\documentclass[12pt]{article}

\usepackage{amsmath}
\usepackage[margin=2.5cm]{geometry}

% If you want to use this package, make sure to download it from the LaTeX
% help page on Quercus (and copy to Overleaf, if using)! Then remove the
% comment indicator % that starts the next line.
% \usepackage{csc}

% Document metadata
\title{CSC165H1 - Problem Set 0}
\author{Xuanqi Wei}
\date{\today}


% Document starts here
\begin{document}
\maketitle
\newpage

\section*{My Courses}
\begin{itemize}
	\item CSC148H1: Intro to Comp Sci by Prof. Diane Horton
	\item CSC165H1: Math Expr \& Rsng for Cs by Prof. Gary Baumgartner
	\item MAT137Y1: Calculus with Proofs by Prof. Clovis Hamel Ascanio
	\item MAT224H1: Linear Algebra 2 by Prof. Denis Gorodkov
	\item STA130H1: STAT Reasoning by Prof. Joshua Speagle
\end{itemize}


\section*{Set notation}

\[
S_1 \cap S_2 = \{0, 1, 4, 5, 6, 9, 10, 11, 14\}
\]


\section*{A truth table}

\begin{table}[h]
\centering

\begin{tabular}{|c|c|c|c|} \hline

	p & q & r & $(p\lor q)\Rightarrow (p\Leftrightarrow r)$ \\
\hline
	T & T & T & T \\
\hline
	T & T & F & F \\
\hline
	T & F & T & T \\
\hline
	T & F & F & F \\
\hline
	F & T & T & F \\
\hline
	F & T & F & T \\
\hline
	F & F & T & T \\
\hline
	F & F & F & T \\ \hline
	
\end{tabular}
\end{table}

\newpage
\section*{A calculation}
\paragraph{Part 1}
\begin{align*}
    \sum_{i=0}^{n-1} (2i + 3) &= 3n  + \frac{2n(n-1)}{2} \\
    &= 3n + n^2 - n \\
    &= n^2 + 2n
\end{align*}

\paragraph{Part 2} 
According to the result of Part 1:

\begin{align*}
	\sum_{i=0}^{n-1} (2i + 3) &\geq 165 \\
	n^2 + 2n &\geq 165 \\
	n^2 + 2n -165 &\geq 0
\end{align*}

According to the quadratic formula, when $n^2 + 2n -165 = 0 $, gives:
$$n=\frac{-b\pm \sqrt{b^2-4ac}}{2a},\ where\ a=1,\ b=2,\ c=-165$$

Therefore,
\begin{align*}
	n&=\frac{-2\pm \sqrt{(-2)^2-4\times1\times(-165)}}{2\times1} \\
	n&=\frac{(-1)\pm 2\sqrt{166}}{2} \\
	n&=-1\pm\sqrt{166}
\end{align*}

Since $a=1>0$, when $n\leq-1-\sqrt{166}\ or\ n\geq-1+\sqrt{166}$, $n^2 + 2n -165 \geq 0$

Since $n\geq0$, when $n\geq-1+\sqrt{166}\approx11.88 $, $n^2 + 2n -165 \geq 0$, which,
$$n = 12$$


\end{document}

\documentclass{article}

\usepackage{enumerate}
\usepackage{graphicx}
\usepackage{geometry}
\usepackage{float}
\usepackage{appendix}
\usepackage{color}
\usepackage{amsmath}
\usepackage{amssymb}
\usepackage{amsfonts}
\usepackage{pifont}
\usepackage{fancyhdr}
\fancyhf{}


\geometry{a4paper}
\geometry{left = 2cm}
\geometry{right = 2cm}
\geometry{top = 3cm}
\geometry{bottom = 3cm}

\pagestyle{fancy}
\rhead{MAT223}
\cfoot{\thepage}

\title{Reflective Writing Assignment 2}

\author{Xuanqi Wei}

\date{\today}

\begin{document}


\maketitle
\thispagestyle{empty}

\newpage

\tableofcontents
\thispagestyle{empty}

\newpage

\setcounter{page}{1}

\section{Mathematical Theme 1: Sets \& Vectors}
\subsection{Reason of Choosing}
\quad The reason of choosing sets, vectors and notations is they are the basic of basic and is our first impression towards linear algebra world. Without knowing clearly how to write and read the sets, all our theorems can't be describe in mathematical language, without knowing clearly what's a vectors and read the vectors, all our linear algebra can be introduced. Sets and vectors in linear algebra are like water and electricity, the basic living condition in Toronto. 

\subsection{Sets}
\quad Set is an unordered collection of distinct objects. There are some important notations:
\begin{enumerate}
	\item $a \in A$: a is an element of set A
	\item $a \notin A $: a in not an element of set A
	\item $\emptyset$: the set containing no elements
	\item $A \subseteq B$: set A is a subset of set B(for all $a /in A$ we also have $a \in B$
	\item $A = B$: $A \subseteq B$ \& $B \subseteq A$
	\item Set-builder Notation: $Y = \{a \in X|some\ rule\ involving\ a \}$
\end{enumerate}

\quad Moreover, there are some common sets:
\begin{enumerate}
	\item $\mathbb{N}: \{Natural\ Numbers\}$
	\item $\mathbb{Z}: \{Integers\}$
	\item $\mathbb{Q}: \{Quotient\ of\ Integers\}$
	\item $\mathbb{R}: \{Real\ Numbers\}$
	\item $\mathbb{R^n}: \{Vectors\ in\ n-dimensional\ Euclidean\ Space\}$
\end{enumerate}

\quad Besides, in this section, the difference between for all and for some is also worth noting.
\begin{enumerate}
	\item $A = \{x|x=2m\ for\ all\ n\in N\}: A = \emptyset$
	\item $B = \{y|y=2m\ for\ some\ n\in N\}: B = set\ of\ all\ even\ number$
\end{enumerate}

\subsection{Vectors}
\noindent Vector is modelling a relationship between points (displacement with a direction and a magnitude). Similarly, there are some notations.
\begin{enumerate}
	\item $\overrightarrow{PQ}$: the vector from P to Q
	\item $|a|$: the magnitude of the vector $\overrightarrow{a}$(normal or length is sometimes called).
	\item $\overrightarrow{0}$: the vector with no magnitude
\end{enumerate}

\noindent In vector, the are the laws of vector arithmetic. The following is the basic operation between vectors $\overrightarrow{v_1}, \overrightarrow{v_2}, ..., \overrightarrow{v_n}$, called Linear Combination: $$ \overrightarrow{w} = a_1\overrightarrow{v_1} + a_2\overrightarrow{v_2} + ... + a_n\overrightarrow{v_n}$$

\newpage

\section{Mathematical Theme 2: System of Linear Equations}
\subsection{Reason of Choosing}
\quad The reason of choosing System of Linear Equations as the second theme is calculations using matrix are frequently being used in the future units. 

\subsection{Linear System}
\quad Linear System is a collection of one or more linear equations involving the same variables. $$\left \{ \begin{array}{rcl}
	a_{11} x_1 + a_{12} x_2 + ... + a_{1n} x_n = b_1 \\
	a_{21} x_1 + a_{22} x_2 + ... + a_{2n} x_n = b_2 \\ 
\\
\\
\\
	a_{n1} x_1 + a_{n2} x_2 + ... + a_{nn} x_n = b_n \end{array}\right. $$
Every row, $a_{11} x_1 + a_{12} x_2 + ... + a_{1n} x_n = b_1$, is a linear equation. There are m equations and n variables. 

A linear system is either consistent which means exactly one solution or infinite many solutions or inconsistent which means no solutions.

\subsection{Solving Matrix}
$$\left \{ \begin{array}{rcl}
	x_2+2x_2+x_3+x_4+x_5 = 2 \\
	2x_1 + x_2 + 3x_3 + 5x_4 + 5x_5 = 7 \\ 
	3x_1 + 6x_2 + 4x_3 + 9x_4 + 10x_5 = 11\\
	x_1 + 2x_2 + 4x_3 + 3x_4 +6x_5 = 9 \end{array}\right. $$
	
Firstly we need to write the argumented matrix for the linear system.
$$\left \{ \begin{array}{rcl}
	x_2+2x_2+x_3+x_4+x_5 = 2 \\
	2x_1 + x_2 + 3x_3 + 5x_4 + 5x_5 = 7 \\ 
	3x_1 + 6x_2 + 4x_3 + 9x_4 + 10x_5 = 11\\
	x_1 + 2x_2 + 4x_3 + 3x_4 +6x_5 = 9 \end{array}\right. \Longrightarrow
\left[\begin{array}{ccccc|r}
	1 & 2 & 1 & 2 & 1 & 2 \\
	2 & 1 & 3 & 5 & 5 & 7 \\
	3 & 6 & 4 & 9 & 10 & 11 \\
	1 & 2 & 4 & 3 & 6 & 9 \\
	\end{array} \right ]$$

Use row reduction to change the matrix into REF and RREF.
$$\xrightarrow[r_4-r_1\\ r_4-3r_3]{r_2-2r_1\\ r_3-3r_1}
\left[\begin{array}{ccccc|r}
	1 & 2 & 1 & 2 & 1 & 2 \\
	0 & -3 & 1 & 1 & 3 & 3 \\
	0 & 0 & 1 & 3 & 7 & 5 \\
	0 & 0 & 0 & -8 & -16 & 8 \\
	\end{array} \right ]$$
$$\xrightarrow[r_2 + \frac{1}{3}r_4\\ r_1-2r_2\\ r_1-r_3\\ r_1-2r_4]{r_4\times\frac{-1}{3}\\ r_4\times\frac{-1}{8} \\ r_3-3r_4 \\ r_2+\frac{1}{3}r_3}
\left[\begin{array}{ccccc|r}
	1 & 0 & 0 & 0 & -4 & -2 \\
	0 & 1 & 0 & 0 & 0 & 0 \\
	0 & 0 & 1 & 0 & 1 & 2 \\
	0 & 0 & 0 & 1 & 2 & 1 \\
	\end{array} \right ]
$$

Use the free variable to express the basic variable and write it in vector form
$$\left \{ \begin{array}{rcl}
	x_1 -4 x_5 = -2 \\
	x_2 = 0 \\ 
	x_3 + x_5 = 2\\
	x_4 + 2 x_5 = 1 \end{array}\right. 
\Longrightarrow
\left \{ \begin{array}{rcl}
	x_1 = 4x_5 - 2 \\
	x_2 = 0\\ 
	x_3 = -x_5 +2\\
	x_4 = -2x_5 +1 \end{array}\right. 
$$
$$\left[\begin{array}{c}
	x_1 \\
	x_2 \\
	x_3 \\
	x_4 \\
	x_5
	\end{array} \right ] =
	x_5 \left[\begin{array}{c}
	4 \\
	0 \\
	-1 \\
	-2 \\
	1
	\end{array} \right ] + 
	\left[\begin{array}{c}
	-2 \\
	0 \\
	2 \\
	1 \\
	0
	\end{array} \right ]$$

\newpage
\section{Mathematical Theme 3: Lines}
\subsection{Reason of Choosing}
\quad The reason of choosing as the third theme is line is also the important knowledge point to be understood and used. The point moves draw a line, which is another form of expressing the line. The line moves draw a plane, which is studied in the future units. 

\subsection{Vector Form of a Line}
\quad As mentioned in the first paragraph, 'point moves draw a line' is the idea of another form of stating a line. $Let\ l\ be\ a\ line\ and\ let\ \overrightarrow{d}\ and\ \overrightarrow{p}\ be\ vectors.$
$$If\ l = \{\overrightarrow{x}|\overrightarrow{x} = t \overrightarrow{d}+\overrightarrow{p}\ for\ some\ t \in \mathbb{R}\},\ we\ say\ the\ vector\ equation:$$
$$\overrightarrow{x} = t\overrightarrow{d} + \overrightarrow{p}\ (where\ t \in \mathbb{R})\ is\ l\ expressed\ in\ vector\ form.$$
The vector $\overrightarrow{d}$ is called a direction vector for l.

\subsection{Determine the relationship between lines}
\subsubsection{Question}
Determine if the lines $l_1$ and $l_2$, given in vector form as: 
$$\overrightarrow{x} = t	\left[\begin{array}{c}
	1 \\
	1
	\end{array} \right ] + 	\left[\begin{array}{c}
	2 \\
	1 
	\end{array} \right ]
$$ and $$\overrightarrow{x} = t	\left[\begin{array}{c}
	2 \\
	2 
	\end{array} \right ] + 	\left[\begin{array}{c}
	4 \\
	3
	\end{array} \right ]
$$ are the same line.

\subsubsection{Answer}
Firstly, give different parametric variables different names
$$If\ \overrightarrow{x} \in l_1,\ \overrightarrow{x} = t \left[\begin{array}{c}
	1 \\
	1
	\end{array} \right ] + 	\left[\begin{array}{c}
	2 \\
	1 
	\end{array} \right ],\ where\ t \in \mathbb{R}$$
$$If\ \overrightarrow{x} \in l_2,\ \overrightarrow{x} = s \left[\begin{array}{c}
	2 \\
	2
	\end{array} \right ] + 	\left[\begin{array}{c}
	4 \\
	3 
	\end{array} \right ],\ where\ s \in \mathbb{R}$$

\noindent Set their equations equal and solve
$$\overrightarrow{x} = t \left[\begin{array}{c}
	1 \\
	1
	\end{array} \right ] + 	\left[\begin{array}{c}
	2 \\
	1 
	\end{array} \right ] = \overrightarrow{x} = \overrightarrow{x} = s \left[\begin{array}{c}
	2 \\
	2
	\end{array} \right ] + 	\left[\begin{array}{c}
	4 \\
	3 
	\end{array} \right ]$$

$$\Longrightarrow \left[\begin{array}{c}
	t + 2 \\
	t + 1
	\end{array} \right ] = \left[\begin{array}{c}
	2s + 4 \\
	2s + 3
	\end{array} \right ] $$
$$\Longrightarrow 0 = 2s - t + 2 $$
This equation has a solution whenever $0 = 2s - t + 2 $ has a solution. $l_1$ = $l_2$. 


\end{document}
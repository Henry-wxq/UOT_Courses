\documentclass[fontsize=12pt]{scrartcl}
\usepackage[utf8]{inputenc}
\usepackage{latexsym, amsfonts, fullpage, lscape, cancel, array, lastpage}
\usepackage{mathrsfs, graphicx, amssymb, amsmath, amscd, amsthm, MnSymbol}
\usepackage{titlesec}
\usepackage{enumerate}
\usepackage{graphicx}
\usepackage{geometry}
\usepackage{float}
\usepackage{appendix}
\usepackage{color}
\usepackage{amsmath}
\usepackage{pifont}
\usepackage{fancyhdr}
\usepackage{enumitem}
\usepackage{amssymb}

\titleformat{\section}
  {\normalfont\Large\bfseries}{\S\thesection}{1em}{}

\usepackage[english]{babel}
\usepackage{eucal}

\title{Unit 6(a) Lecture Notes for MAT224}
\author{Xuanqi Wei 1009353209}

\date{28-30 March 2023}

\begin{document}

\maketitle

\newpage

\section{6.3 Jordan Canonical Form}

\bigskip

\noindent
\textbf{Definition (6.3.2)} Let $T: V \to V$ be a linear mapping on a finite-dimensional vector space $V$. Let $\lambda$ be an eigenvalue of $T$ with algebraic multiplicity $m$.\\
\\
(a) The \textbf{$\lambda$-generalized eigenspace}, written $K_{\lambda}$, is the kernel of the mapping $(T - \lambda I)^m$ on $V$.\\
\\
(b) The non-zero elements of $K_{\lambda}$ are called \textbf{generalized eigenvectors} of $T$.\\
\bigskip

\hline
\bigskip

\noindent
What is the dimension of the generalized eigenspace $K_2$ with respect to the standard basis $\mathcal{E}$, where $[T]_{\mathcal{E}}^{\mathcal{E}}$ is the following matrix?

$$\begin{bmatrix} 2 & 1 & 0 & 0 & 0 & 0 \\
0 & 2 & 1 & 0 & 0 & 0 \\
0 & 0 & 2 & 0 & 0 & 0 \\
0 & 0 & 0 & 3 & 0 & 0 \\
0 & 0 & 0 & 0 & 4 & 1 \\
0 & 0 & 0 & 0 & 0 & 4 \end{bmatrix} $$

To find the generalized eigenspace $K_2$, we need to find the nullspace of $(T - 2I)^2$. We can calculate $(T - 2I)^2$ as follows:

$$
\begin{bmatrix}
0 & 0 & 1 & 0 & 0 & 0 \\
0 & 0 & 0 & 0 & 0 & 0 \\
0 & 0 & 0 & 1 & 0 & 0 \\
0 & 0 & 0 & 0 & 2 & 1 \\
0 & 0 & 0 & 0 & 0 & 2 \end{bmatrix} \begin{bmatrix} 0 & 1 & 0 & 0 & 0 & 0 \\
0 & 0 & 1 & 0 & 0 & 0 \\
0 & 0 & 0 & 0 & 0 & 0 \\
0 & 0 & 0 & 1 & 0 & 0 \\
0 & 0 & 0 & 0 & 2 & 1 \\
0 & 0 & 0 & 0 & 0 & 2 \end{bmatrix} = \begin{bmatrix} 0 & 0 & 1 & 0 & 0 & 0 \\
0 & 0 & 0 & 0 & 0 & 0 \\
0 & 0 & 0 & 0 & 0 & 0 \\
0 & 0 & 0 & 0 & 0 & 0 \\
0 & 0 & 0 & 0 & 0 & 1 \\
0 & 0 & 0 & 0 & 0 & 0 \end{bmatrix}.$$
So we need to find the nullspace of this matrix. We can do this using row reduction:
$$\begin{bmatrix} 0 & 0 & 1 & 0 & 0 & 0 \\
0 & 0 & 0 & 0 & 0 & 0 \\
0 & 0 & 0 & 0 & 0 & 0 \\
0 & 0 & 0 & 0 & 0 & 0 \\
0 & 0 & 0 & 0 & 0 & 1 \\
0 & 0 & 0 & 0 & 0 & 0 \end{bmatrix} \sim \begin{bmatrix} 0 & 0 & 1 & 0 & 0 & 0 \\
0 & 0 & 0 & 0 & 0 & 1 \\
0 & 0 & 0 & 0 & 0 & 0 \\
0 & 0 & 0 & 0 & 0 & 0 \\
0 & 0 & 0 & 0 & 0 & 0 \\
0 & 0 & 0 & 0 & 0 & 0 \end{bmatrix}.$$
The last row indicates that $x_6$ is a free variable, so the nullspace has dimension 1. Therefore, the dimension of the generalized eigenspace $K_2$ is 1.

\noindent
Consider a vector $v$ with the property that $Tv = 2v$. This implies $Tv - 2v = 0$.\\ 
So $(T-2I)v=$ means that $v$ is an eigenvector for $\labmda = 2$.\\
\\
Now consider vector $w$ such that $(T-2I)^3w = 0$. We can see that 3 is the\\
multiplicity of $(2-\lambda)$ in the characteristic polynomial of $T$. In other words, 3 is the algebraic multiplicity of $\lambda = 2$.\\
\\
If $(T - 2I)^3w = 0$, then $(T-2I)(T-2I)^2w = 0$. So $(T-2I)^2w$ is an eigenvector.\\
\\
Also, $(T-2I)^2(T-2I)w = 0$. So $u = (T-2I)w$ satisfies the equation:\\
$$(T-2I)^2u = 0$$

\medskip

\noindent
Is it possible to write a cycle of 3 vectors $\{v, (T-2I)v, (T-2I)^2v\}$ as a basis for the $K_2$ from above?



To determine if the cycle ${v, (T-2I)v, (T-2I)^2v}$ can be a basis for the generalized eigenspace $K_2$ of the matrix $[T]_{\mathcal{E}}^{\mathcal{E}}$, where $\mathcal{E}$ is the standard basis, we need to check if it satisfies the properties of a cycle basis as defined in Section 6.2.
First, we need to verify that $(T-2I)^2v$ is not a scalar multiple of $(T-2I)v$. We have

$$(T-2I)((T-2I)v) = (T-2I)^2v = 0$$

which implies that $(T-2I)v$ is an eigenvector of $T$ with eigenvalue $\lambda = 2$. Since the eigenspace of $\lambda = 2$ has dimension 3, we know that there is at least one more linearly independent eigenvector corresponding to $\lambda = 2$. Therefore, $(T-2I)^2v$ is not a scalar multiple of $(T-2I)v$, and the cycle ${v, (T-2I)v, (T-2I)^2v}$ satisfies the first property of a cycle basis.
Next, we need to verify that $(T-2I)^3v = 0$, which implies that the cycle is minimal. We have
$$(T-2I)^3v = (T-2I)((T-2I)^2v) = (T-2I)(T-2I)^2v = (T-2I)0 = 0$$
Therefore, the cycle is minimal.
Finally, we need to verify that the cycle is linearly independent. Suppose there exist scalars $a,b,c$ such that
$$av + b(T-2I)v + c(T-2I)^2v = 0$$

Then we have

$$(T-2I)(av + b(T-2I)v + c(T-2I)^2v) = 0$$

which simplifies to

$$(T-2I)(T-2I)(av + b(T-2I)v) = 0$$

Since $(T-2I)^2v$ is not a scalar multiple of $(T-2I)v$, we know that $av + b(T-2I)v$ is not an eigenvector corresponding to $\lambda = 2$. Therefore, we must have $av + b(T-2I)v = 0$, which implies that $a = b = 0$. Substituting this into the original equation gives $c(T-2I)^2v = 0$, which implies that $c = 0$. Therefore, the cycle ${v, (T-2I)v, (T-2I)^2v}$ is linearly independent.
Since the cycle satisfies all the properties of a cycle basis, it is a basis for the generalized eigenspace $K_2$.

\section{6.4 Computing Jordan Form}

\bigskip

\noindent
What is Jordan Canonical Form and how can you figure out what it will look like for a transformation $T$ without calculating a canonical basis?\\

\noindent
Jordan Canonical Form is a way of writing a matrix that represents a linear transformation $T: V \to V$ in a specific, simplified form. The Jordan Canonical Form of a matrix is essentially a block diagonal matrix where each block represents a Jordan block corresponding to a distinct eigenvalue of $T$.
\\


We will look at Example 6.4.3 together. Then you will write the Jordan canonical form for the transformation in Example 6.4.4.\\
\\
\textbf{Example 6.4.3} Let $\beta = \{e^x, xe^x, x^2 e^x , x^3 e^x , e^{-x}, xe^{-x}\}$ and let $V = \span\{\beta \}$. Let $T : V \to V$ be defined by $T(f) = 2f + f^{\prime \prime}$. In order to find the Jordan canonical of $T$, we will first write $[T]_{\beta}^{\beta}$. For clarities sake, I will rename it $M$. So $M = [T]_{\beta}^{\beta}$. Then

$$M = \begin{bmatrix} 3 & 2 & 2 & 0 & 0 & 0 \\
0 & 3 & 4 & 6 & 0 & 0\\
0 & 0 & 3 & 6 & 0 & 0\\
0 & 0 & 0 & 3 & 0 & 0\\
0 & 0 & 0 & 0 & 3 & -2\\
0 & 0 & 0 & 0 & 0 & 3\end{bmatrix}
$$

\noindent
Finding the Jordan canonical form (JCF) of a transformation requires similar calculations to diagonalizing a diagonalizable matrix. We take the characteristic polynomial and find that it is $(3-\lambda)^6$. If the dimension of the eigenspace for $\lambda = 3$ was also 6, it would be diagonalizable.\\
\\
However, $\dim(E_3) = 2$. We can see this by looking at

$$(M- 3I)v = 0 \ \implies \  \begin{bmatrix} 0 & 2 & 2 & 0 & 0 & 0 \\
0 & 0 & 4 & 6 & 0 & 0\\
0 & 0 & 0 & 6 & 0 & 0\\
0 & 0 & 0 & 0 & 0 & 0\\
0 & 0 & 0 & 0 & 0 & -2\\
0 & 0 & 0 & 0 & 0 & 0\end{bmatrix} \begin{bmatrix} x_1 \\ x_2 \\ x_3 \\ x_4 \\ x_5 \\ x_6 \end{bmatrix} = \begin{bmatrix}  2x_2 + x_3 \\  4 x_3 + 6 x_4 \\  6x_4 \\ 0 \\  - 2x_6 \\ 0 \end{bmatrix} = \begin{bmatrix} 0 \\ 0 \\ 0 \\ 0 \\ 0 \\ 0 \end{bmatrix} $$

\noindent
So $x_1$ and $x_5$ can be anything. Therefore the matrix representation of the vectors spanning $E_3$ are $\begin{pmatrix} 1 \\ 0 \\ 0 \\ 0 \\ 0 \\ 0 \end{pmatrix} \mbox{ and } \begin{pmatrix} 0 \\ 0 \\ 0 \\ 0 \\ 1 \\ 0 \end{pmatrix} $. The actual basis elements are $\beta_1 = e^x$ and $\beta_5 = e^{-x}$.\\
\\
\\
This show us that the set of solutions $v$ to the equation $(M - 3I)v = 0$ (the set of eigenvectors) is a 2-dimensional space.\\
\\
Another way to say this is:\\
$\dim(\ker(M - 3I)) = 2$.\\
\\
We can also calculate:\\
$\dim(\ker(M - 3I)^2)$\\
$\dim(\ker(M - 3I)^3)$\\
\ \ \ \ \ $\vdots$\\
$\dim(\ker(M - 3I)^6)$ because 6 is the algebraic multiplicity of $\lambda = 3$ (see the characteristic polynomial).\\
\\
But it turns out we don't even have to look that far, since $\dim(\ker(M - 3I)^4) = 6$. So the kernel of $(M - 3I)^4$ is all of $V$.\\
\\
Notice how $(M - 3I)v = 0$ implies that $((M - 3I)^2)v = (M-3I)0 = 0$\\
\\
Similarly, all solutions to $((M - 3I)^{i})v = 0$ are solutions to $((M - 3I)^{i+1})v = 0$.\\
\\
That is why we look at the values of $\dim(\ker(M - 3I)^{i+1}) - \dim(\ker(M-3I)^{i})$.\\
We are looking to see how many basis elements of $\ker((M - 3I)^{i+1})$ are not also elements of $\ker((M - 3I)^{i})$.\\
\\
Consider a vector that is in $\ker(M - 3I)^4)$ , but not $\ker(M - 3I)^3$. Let's call it $w$. Then $(M - 3I)w$, $((M-3I)^2) w$, and $((M-3I)^3 )w$ are all non-zero and linearly independent from each other (by theorem).\\
\\
Let $S_1 = \{ w, (M-3I)w, ((M-3I)^2) w, ((M-3I)^3) w\}$. Then $S_1$ is a basis for a 4-dimensional $(M-3I)$-invariant subspace. It is also a cycle of length 4.\\
\\
But $\dim(\ker((M-3I)^2)) = 4$. Compare that to the fact that $\dim(\ker(M-3I) = 2$. So there are 4-2 = 2 extra basis elements of $\ker(M-3I)^2$ than there are basis elements of $E_3$. This means we can create two non-overlapping cycles from basis elements of $\ker(M-3I)^2$ that are not eigenvectors.\\
\\
This information can be summarized with this tableau for eigenvalue $\lambda = 3$:

\includegraphics{diag1.png}

\noindent
The number of squares in the first column is equal to the dimension of $E_3$, which is 2.\\
\\
The number of squares in the second column is equal to\\
\\
$\dim(\ker(M - 3I)^2) - \dim(\ker(M-3I)) = 4 - 2 = 2$.\\
\\
The number of squares in the third column is equal to \\
$\dim(\ker(M - 3I)^3) - \dim(\ker(M-3I)^2) = 5-4 = 1$\\
\\
The number of squares in the last column is equal to\\
$\dim(\ker(M - 3I)^4) - \dim(\ker(M-3I)^3) = 6-5 = 1$\\
\\
From this we can see that there are two non-overlapping cycles that general $(M-3I)$-invariant subspaces, one of length four corresponding to the top row, and one of length two corresponding to the second (bottom) row.\\
\\
So $M-3I$ is a nilpotent matrix with zeros on the diagonal and two Jordan blocks, one that is $4\times 4$ and the other is $2\times 2$.\\
\\
Instead of writing the matrix for $M-3I$, we will add the 3's back to the main diagonal and write the Jordan canonical form of $T$:

$$\begin{bmatrix} 3 & 1 & 0 & 0\\
0 & 3 & 1 & 0\\
0 & 0 & 3 & 1\\
0 & 0 & 0 & 3\\
  & & & & 3 & 1 \\
  & & & & 0 & 3
  \end{bmatrix}
$$

\bigskip

\noindent
Now you try! Below is the matrix featured in Example 6.4.4. Find it's Jordan canoncial form by creating tableau diagrams for each of its eigenvalues. You can check your answer against the textbook's on pages 285 and 286.

$$A = \left[ \begin{array}{c c c c c} 1-i & 1 & 0 & 0 & 0 \\
0 & 1-i & 0 & 0 & 0 \\
i & -1 & 1 & 0 & 0\\
0 & i & 1 & 1 & 1\\
-i & 1 & 0 & 0 & 1
\end{array} \right]$$

\bigskip

\noindent
Hint: The characteristic polynomial of $A$ is $(1-\lambda)^3((1-i)-\lambda )^2. $ Therefore, you should do two cycle tableaus: one for $\lambda = 1$ and the other for $\lambda = 1-i.$\\

To find the Jordan canonical form of $A$, we first need to find all its eigenvalues and their corresponding eigenvectors. Let's compute the characteristic polynomial of $A$:







\end{document}
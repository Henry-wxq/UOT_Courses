\documentclass[12pt]{article}

\usepackage{enumerate}
\usepackage{graphicx}
\usepackage{geometry}
\usepackage{float}
\usepackage{appendix}
\usepackage{color}
\usepackage{amsmath}
\usepackage{pifont}
\usepackage{fancyhdr}
\usepackage{enumitem}
\usepackage{amssymb}
\fancyhf{}


\geometry{a4paper}
\geometry{left = 2cm}
\geometry{right = 2cm}
\geometry{top = 3cm}
\geometry{bottom = 3cm}

\pagestyle{fancy}
\lhead{Fall 2023-2024}
\rhead{MAT246 Online Quiz 1}
\cfoot{\thepage}

\title{MAT246 Online Quiz 1}

\author{Xuanqi Wei 1009353209}

\date{\today}

\begin{document}

\maketitle
\thispagestyle{empty}

\newpage

\setcounter{page}{1}

\section{Question 1}

After Carefully reading the document `Course objectives', two of the objectives that appeal to me which I'm aware of and should make efforts to achieve are the Course Content and Presentation/Communication.

\noindent According to the Course Objective Document, Course Content including understanding abstract concepts, knowing why they are introduced, and to integrate them into one's thinking process as tools for thinking. After briefly reading though the content of our course book, I found that the famous Fermat’s Theorem is included to be taught in our course which excites me a lot. These knowledge of number theory will significantly broaden my horizon and take me into more details in mathematics. For improvement of this part, I plan to understand every posted puzzles at the intuitive level. Moreover, I will do my best to understand and memory exact details of definitions and statements of theorems, knowing how to apply these learnt definitions and results after the course. Finally, I will use the learnt concepts when conducting problem solving and composing proofs.

\noindent As for the Presentation/Communication part, I’m personally an abroad student which English is not my first language. I will do my best to write every needed essays and do my best to stick to the formal format of Problem sets and Quizzes. I will try to further improve myself in academic discussion by discussing MAT246 mathematical problem with my friends. I strongly believe that with these improving path, I will develop my skills a lot and finally achieve my goal at the end of the semester.

\section{Question 2}

After reading the document `Types of Knowledge' and the related documents, I self-recognized I'm strongest in Active, while intuitive being the second and passive to be the last.

\noindent As mentioned in the documents, Active is about `analytical/proof', which must be placed within a formal framework, where is the most important point why I choose to study mathematics as my major. I'm strongly interested in formal mathematical proof starting especially in my high school. I love reading those rigorous proofs in the paper in which I always feel a sense of achievement after following their thoughts. Always, after knowing a new theorem, I extremely want to sketch the process of proofing it before using it. As for the intuitive, I always want to think of a problem myself before searching and discussing with my friends. I can always come up an un-structured thought from my mind which sometimes does benefit the following steps but sometimes doesn't, so I listed it as the second. For the passive knowledge, as I mentioned before, I always want to see the rigorous proof before using a theorem which made me hard to simply remember a documented fact without understanding it. That’s the reason why I list it in the third place.

\newpage

\section{Question 3}
Question: Prove that for a given natural number n, the successor of n, S(n), or n+1, must be unique. That is, there can't be two distinct successors for n.

\noindent \textbf{Proof}: 

\noindent Let $n \in \mathbb{N}$. Let S(n) be the successor of n, which $S_1(n) = n+1$.

\noindent Assume the successor for n, $S_1(n)=n+1$ is not unique, which, $$\exists m \in \mathbb{N},\ S_2(n)=m,\ and\ m\neq n+1$$

\noindent According to the property of natural number, since $n \in \mathbb{N}$, gives, $(n + 1) \in \mathbb{N}$.

\noindent According to the Trichotomy Principle in page 11, since $(n+1)\in \mathbb{N},\ m \in \mathbb{N}$, and $m\neq (n+1)$, gives, 
$$m < (n+1)\ or\ m > (n+1)$$

\ding{172} $m < (n+1)$

Since $S_2(n)=m$, gives $n < m$, which,
$$n<m<(n+1)$$

Since $m \in \mathbb{N}$, this contradicts property 2 of natural numbers. So the option $m < (n+1)$ is not possible.

\ding{173} $m > (n+1)$

Since $S_1(n) = n+1$, gives $n < (n+1)$, which, 
$$n<(n+1)<m$$

Therefore, $\exists k \in \mathbb{N}, k = m - (n+1)$, which $S_1(n)=n+1$, $S_2(n)=(n+1)+k$
\begin{align*}
    n+1 &< (n+1)+k \\
    0 &< k
\end{align*}


Since when $S_2$ is the successor of n, $k=0$. However, this contradicts to $0<k$. So the option $m > (n+1)$ is not possible.

\noindent Therefore, this means $S_1(n) = S_2(n),\ where\ n \in \mathbb{N}$, which there can
t be two distinct successors for n.

\quad \quad \quad \quad \quad \quad \quad \quad \quad \quad \quad \quad \quad \quad \quad \quad \quad \quad \quad \quad \quad \quad \quad \quad \quad \quad \quad \quad \quad \quad \quad \quad \quad \quad \quad \quad \quad \quad \quad $\blacksquare$



\end{document}

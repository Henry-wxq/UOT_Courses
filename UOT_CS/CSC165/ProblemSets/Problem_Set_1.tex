\documentclass[12pt]{article}

\usepackage{enumerate}
\usepackage{graphicx}
\usepackage{geometry}
\usepackage{float}
\usepackage{appendix}
\usepackage{color}
\usepackage{amsmath}
\usepackage{pifont}
\usepackage{fancyhdr}
\usepackage{enumitem}
\usepackage{amssymb}
\fancyhf{}


\geometry{a4paper}
\geometry{left = 2cm}
\geometry{right = 2cm}
\geometry{top = 3cm}
\geometry{bottom = 3cm}

\pagestyle{fancy}
\lhead{CSC165H1, Winter 2023}
\rhead{Problem Set 1}
\cfoot{\thepage}

\title{CSC165 Problem Set 1}

\author{Jonah Kisluk, Nick Pestov, Xuanqi Wei}

\date{\today}

\begin{document}


\maketitle
\thispagestyle{empty}

\newpage

\tableofcontents
\thispagestyle{empty}

\newpage

\setcounter{page}{1}

\section{Question 1: Propositional Formulas}
\begin{enumerate}
	\item $(\neg p \Leftrightarrow q)\Rightarrow q $
	\begin{enumerate}[label=(\roman*)]
	\setcounter{enumii}{1}

\begin{table}[h]
\centering
\begin{tabular}{|c|c|c|} \hline
p & q & $(\neg p \Leftrightarrow q)\Rightarrow q $ \\
\hline
	T & T & T \\
\hline
	T & F & F \\
\hline
	F & T & T \\
\hline
	F & F & T \\ \hline
\end{tabular}
\caption{(\romannumeral1)}
\end{table}
		\item Equivalent Formula: $(\neg p \wedge \neg q)\vee(q\wedge p)\vee q$ 
\begin{align*}
	&(\neg p \wedge \neg q)\vee(q\wedge p)\vee q \\
	\equiv &\neg((\neg \neg p\vee q)\wedge \neg(\neg q\vee \neg p)\vee q) \ (De\ Morgan's)(Double\ Negation)\\
	\equiv &\neg(\neg p\Rightarrow q) \vee \neg(q \Rightarrow \neg p)\vee q\ (Implication)\\
	\equiv &\neg((\neg p \Rightarrow q)\wedge(q \Rightarrow \neg p)) \vee q\ (De Morgan's)\\
	\equiv &\neg(\neg p \Leftrightarrow q) \vee q\ (Bi-Implication)\\
	\equiv &(\neg p \Leftrightarrow q) \Rightarrow q\ (Implication)
\end{align*}
	\end{enumerate}
		\item $(p\Rightarrow (q\Rightarrow r))\Rightarrow ((p\Rightarrow q)\Rightarrow r) $
	\begin{enumerate}[label=(\roman*)]
	\setcounter{enumii}{1}

\begin{table}[h]
\centering
\begin{tabular}{|c|c|c|c|} \hline
p & q & r & $(p\Rightarrow (q\Rightarrow r))\Rightarrow ((p\Rightarrow q)\Rightarrow r) $ \\
\hline
	T & T & T & T \\
\hline
	T & T & F & T \\
\hline
	T & F & T & T \\
\hline
	T & F & F & T \\
\hline
	F & T & T & T \\
\hline
	F & T & F & F \\
\hline
	F & F & T & T \\
\hline
	F & F & F & F \\ \hline
\end{tabular}
\caption{(\romannumeral1)}
\end{table}
		\item Equivalent Formula: $(p\wedge q\wedge \neg r)\vee ((p\wedge \neg q)\vee r)$ 
\begin{align*}
	&(p\wedge q\wedge \neg r)\vee ((p\wedge \neg q)\vee r) \\
	\equiv &(\neg \neg p \wedge \neg (\neg q \vee r))\vee ((\neg \neg p \wedge \neg q) \vee r\ (De\ Morgan's)(Double Negation) \\
	\equiv &\neg(\neg p\vee (\neg q \vee r))\vee (\neg(\neg p \vee q)\vee r)\ (De\ Morgan's)\\
	\equiv &\neg(\neg p \vee (q\Rightarrow r)) \vee (\neg(p \Rightarrow q)\vee r)\ (Implication)\\
	\equiv &\neg(p\Rightarrow (q \Rightarrow r))\vee ((p\Rightarrow q) \Rightarrow r)\ (Implication)\\
	\equiv &(p\Rightarrow (q\Rightarrow r))\Rightarrow ((p\Rightarrow q)\Rightarrow r)\ (Implication)
\end{align*}
	\end{enumerate}
	
\end{enumerate}
\newpage

\section{Question 2: Translating Statements}
\begin{enumerate}[label=(\alph*)]
	\item $\forall c\in C,\ \forall s \in S,\ \neg CS(s)\vee \neg Fail(s, c)$
	\item $\exists c \in C,\ \forall s \in S,\ CS(s)\Rightarrow Study(s, c)$
	\item $\exists s \in S,\ \forall c \in C,\ CS(s) \wedge \neg Study(s, c)$
	\item $\forall c \in C,\ (\exists s_1 \in S,\ Study(s_1, c)\wedge (\forall s_2 \in S,\ Study(s_2, c)\Rightarrow s_1=s_2))$
	\item $\forall s \in S,\ \exists c_1,\ c_2 \in C,\ Study(s, c_1)\wedge Study(s, c_2)\wedge c_1 \neq c_2 \wedge (\forall c_3 \in C,\ Study(s, c_3)\Rightarrow c_3=c_1 \vee\ c_3=c_2)$
\end{enumerate}

\section{Question 3: Choosing a Universe and Predicates}
\begin{enumerate}[label=(\alph*)]
	\item $For\ each\ x, z \in \mathbb{N}\ define\ P(x, z)\ by\ x<z$
\begin{align*}
	&First\ Statement: False. \ Since\ x=166\ makes\ both\ P(x, 165)\ and\ P(x, 0) false.\\
	&Second\ Statement: True.\ First\ Part,\ \forall x \in \mathbb{N},\ P(x, 165),\ is\ false\ since\ not\ all\ all\ natural\\
	&number\ are\ less\ than\ 165.\ Hence,\ expression\ is\ vacuously\ true,\ as\ anticedent\ is\ not\\ 
	&satisfied.\\
	&Hence,\ it\ satisfies\ the\ requirement\ of\ the\ question.
\end{align*}
	\item $For\ each\ x, z \in \mathbb{N}\ define\ P(x, z)\ by\  x\geq z$
\begin{align*}
	&First\ Statement: True. \ Since\ not\ all\ natural\ numbers\ are\ greater\ or\ equal\ to\ 165,\ but\ all\\
	&natural\ numbers\ by\ definition\ are\ greater\ than\ or\ equal\ to\ 0,\ so\ the\ statement\ is\ True.\\
	&Second\ Statement: True.\ Since\ not\ all\ natural\ numbers\ are\ bigger\ or\ equal\ to\ 165,\ for\\ 
	&example\,\ 50\ is\ a\ natural\ number\ but\ it\ is\ smaller\ than\ 165,\ the\ antecedent\ is\ false.\\
	&Therefore,\ no\ matter\ what\ the\ consequent\ is,\ the\ statement\ is\ vacuously\ True. \\
	&Hence,\ it\ satisfies\ the\ requirement\ of\ the \ question.
\end{align*}
	\item $Let\ S:\{1, 2, 3\},\ T:\{2, 3, 4\},\ for\ each\ x\in S,\ y \in T\ define\ P(x, y)\ by\ y>x\ and\ Q(x)\ by\ x>2$
\begin{align*}
	&First\ Statement: False.\ Since\ for\ any\ arbitrary\ number\ x\ in\ S,\ there\  exists\ a\ number\\
	&y=4\ in\ T,\ which\ is\ bigger\ than\ any\ number\ in\ S,\ making\ the\ antecedent\ True.\ However,\\
	&not\ all\ numbers\ in\ S\ are\ bigger\ than\ 2.\ Thus,\ Ture\ implies\ False\ is\ False.\\
	&Second\ Statement: True.\ For\ any\ arbitrary\ number\ x\ in\ S,\ x\ is\ not\ always\ bigger\ than\ 2,\\
	&which\ the\ antecedent\ is\ False.\ Thus,\ False\ implies\ False\ is\ True.\\
	&Hence,\ it\ satisfies\ the\ requirement\ of \ the\ question.
\end{align*}
\end{enumerate}
\newpage

\section{Question 4: Pierre Numbers}
\begin{enumerate}[label=(\alph*)]
	\item $For\ each\ n \in \mathbb{N}\ define\ PierreNumber(n)\ by\ \exists k \in \mathbb{Z}\ s.t.\ n=2^{2^k} + 1$
	\item $\forall n \in \mathbb{N},\ PierreNumber(n)\Rightarrow (\exists j \in \mathbb{Z},\ 2j + 1 = n)$
	\item $\forall n \in \mathbb{N},\ PierreNumber(n) \Rightarrow \exists x \in \mathbb{N},\ x>n\wedge PierreNumber(x)$
	\item $\forall k \in \mathbb{N}, Prime(2^{2^k}+1)\Rightarrow k \leq 4$
\end{enumerate}

\end{document}
\documentclass[12pt]{article}

\usepackage{enumerate}
\usepackage{graphicx}
\usepackage{geometry}
\usepackage{float}
\usepackage{appendix}
\usepackage{color}
\usepackage{amsmath}
\usepackage{pifont}
\usepackage{fancyhdr}
\usepackage{enumitem}
\usepackage{amssymb}
\fancyhf{}


\geometry{a4paper}
\geometry{left = 2cm}
\geometry{right = 2cm}
\geometry{top = 3cm}
\geometry{bottom = 3cm}

\pagestyle{fancy}
\lhead{CSC165H1, Winter 2023}
\rhead{Problem Set 2}
\cfoot{\thepage}

\title{CSC165 Problem Set 2}

\author{Jonah Kisluk, Nick Pestov, Xuanqi Wei}

\date{\today}

\begin{document}


\maketitle
\thispagestyle{empty}

\newpage

\tableofcontents
\thispagestyle{empty}

\newpage

\setcounter{page}{1}

\section{Q1: Number Theory}
\begin{enumerate}[label=(\alph*)]
	\item Fact from the worksheet:
\begin{align*}
	&\forall a, b, \in \mathbb{Z}, a \neq 0 \lor b \neq 0 \Rightarrow (\exists p_1, q_1 \in \mathbb{Z}, gcd(a, b) = p_1a + q_1b) \wedge (\forall d \in \mathbb{Z^+}, (\exists p_2, q_2 \in \mathbb{Z}, d = \\
	&p_2a+q_2b)\Rightarrow d \geq gcd(a, b))
\end{align*}

Proof:

Let $n \in \mathbb{N}$

Since $n \in \mathbb{N},\ n\in \mathbb{Z}\ and\ n\geq 0$, gives, 

$$9n+1 \geq 1\ and\ 10n+1\geq 1; 9n+1\in \mathbb{Z}\ and\ 10n+1 \in \mathbb{Z}$$

Thus, $9n+1 \neq 0$ and $10n+1 \neq 0$.

By the fact form the worksheet(as listed above), we have,
$$\forall d \in \mathbb{Z^+}, (\exists p_2, q_2 \in \mathbb{Z}, d = p_2(9n+1)+q_2(10n+1))\Rightarrow d \geq gcd(9n+1, 10n+1))$$
Let $d_1 = 1$, which $d \in \mathbb{Z^+}$, gives
$$ d_1 = 10(9n+1)+9(10n+1) \text{ and } gcd(9n+1, 10n+1) \leq d_1 = 1$$
According to fact 4 in worksheet 4, gives, 
$$\forall n, m \in \mathbb{N}, n\neq 0 \vee m \neq 0 \Rightarrow gcd(n, m) \geq 1 $$
Since $9n+1 \neq 0$ and $10n+1 \neq 0$, $gcd(9n+1, 10n+1) \geq 1$.

Since $gcd(9n+1, 10n+1) \geq 1$ and $gcd(9n+1, 10n+1) \leq 1$, gives,

$$gcd(9n+1, 10n+1) = 1$$

Using the worksheet fact, and using a = 9n + 1, b = 10n + 1 and d = 1, we can conclude that gcd(9n + 1, 10n + 1) = 1. $\quad \quad \quad \quad \quad \quad \quad \quad \quad \quad \quad \quad \quad \quad \quad \quad \quad \quad \quad \quad \quad \quad \quad \quad \blacksquare $


	\item Proof:

Let $m, n \in \mathbb{Z}$, assume $n\ |\ m\ \wedge Prime(n)$

Given $\exists k_1 \in \mathbb{Z}\ s.t.\ m=k_1n$ according to $n\ | \ m$

Suppose $n \ | \ (m+1)$, then $\exists k_2 \in \mathbb{Z}\ s.t.\ m+1 = k_2n$

Subtract two equations, gives, 
\begin{align*}
	k_2n - k_1n &= (m+1) - m \\
	n(k_2 - k_1) &= 1
\end{align*}
Since $n$ is a prime, $n > 1$; Since $k_1, k_2 \in \mathbb{Z}, (k_2 - k_1) \in \mathbb{Z}$

Therefore, $n \cdot (k_2 - k_1) \neq 1$ (for $n > 1 \text{ but } 0 < \frac{1}{k_2 - k_1}\leq 1 $), gives,

It contradicts to $n(k_2 - k_1) = 1$

Thus, $n \nmid (m + 1)$
$\quad \quad \quad \quad \quad \quad \quad \quad \quad \quad \quad \quad \quad \quad \quad \quad \quad \quad \quad \quad \quad \quad \quad \quad \quad \quad \quad \quad \quad \quad \blacksquare $
\end{enumerate}

\newpage

\section{Q2: Floors and Ceilings}
\begin{enumerate}[label=(\alph*)]
	\item Proof

$Let\ x \in \mathbb{N}.$ We'll separate the proof into two cases: Either x is even or x is odd. 

\textbf{Case 1: Let x be even} 

By definition: $\exists k \in \mathbb{Z}, x = 2k$  (Since\ floor\ functions\ have\ no\ effect\ on\ integers,\ and $\frac{x}{2}$ is an integer)
$$\left \lceil \frac{x-1}{2} \right \rceil 
    = \left \lceil \frac{2k-1}{2} \right\rceil 
    = \left\lceil k - \frac{1}{2} \right \rceil 
    = k 
    = \frac{x}{2} 
    = \left\lfloor \frac{x}{2} \right\rfloor $$
   
 
\textbf{Case 2: Let x be odd} \\
By definition: $\exists k \in \mathbb{Z}, x = 2k - 1$

\begin{align*}
	&\left \lceil \frac{x-1}{2} \right \rceil 
    = \left \lceil \frac{2k-1 - 1}{2} \right\rceil 
    = \left\lceil k - 1 \right \rceil 
    = k - 1 \\
    = &\left\lfloor k - \frac{1}{2} \right\rfloor \text{works since k is an integer} \\
    = &\left\lfloor \frac{2k - 1}{2} \right\rfloor
    = \left\lfloor \frac{x}{2} \right\rfloor 
\end{align*}
$\quad \quad \quad \quad \quad \quad \quad \quad \quad \quad \quad \quad \quad \quad \quad \quad \quad \quad \quad \quad \quad \quad \quad \quad \quad \quad \quad \quad \quad \blacksquare $

\item 
	\begin{enumerate}[label=(\roman*)]
	\setcounter{enumii}{0}
		\item We want to prove $\forall x \in \mathbb{R}, \left \lceil x-1 \right \rceil = \left \lceil x \right \rceil - 1$

Proof

Let $x \in \mathbb{R}$, gives,
\begin{align*}
	x \leq &\left \lceil x \right \rceil < x + 1 \\
	x - 1 \leq &\left \lceil x \right \rceil - 1 < x
\end{align*}
According to the fact that, 
\begin{align*}
	0 \leq &\left \lceil x \right \rceil - x < 1 \\
	0 \leq &\left \lceil x - 1 \right \rceil - (x - 1) < 1 \\
	x - 1 \leq &\left \lceil x - 1 \right \rceil < x
\end{align*}
We want to prove that $\forall x \in \mathbb{R}, \exists n \in \mathbb{Z}, x -1 \leq n < x $.

Let $x \in \mathbb{R}$. Assume there're two integers between x - 1 and x, which, 
$$\exists n_1, n_2 \in \mathbb{Z}, x - 1 \leq n_1 < n_2 < x, \text{which }n_1 \neq n_2$$
Since $n_1 \neq n_2,\ n_1 < n_2$ and $n_1, n_2 \in \mathbb{Z}$, gives, $min(n_2 - n_1) = 1$

Since $n_2<x$, gives, $x-n_2>0$

Thus, $min(x - n_1) > 1$, gives, $min(x - (x-1)) > 1$

However, x - (x - 1) = 1, which contradicts to $min(x - (x-1)) > 1$.

Therefore, we've proved that $\forall x \in \mathbb{R}, \exists n \in \mathbb{Z}, x -1 \leq n < x $

Since $\left \lceil x \right \rceil - 1 \in \mathbb{Z}$ and $x - 1 \leq \left \lceil x \right \rceil - 1 < x$, gives, $\left \lceil x \right \rceil - 1$ is the only integer between x and $x - 1$.

Since $\left \lceil x - 1 \right \rceil \in \mathbb{Z}$ and $x - 1 \leq \left \lceil x - 1 \right \rceil < x $, gives, $\left \lceil x - 1 \right \rceil$ is the only integer between x and $x - 1$

Therefore, we have $\forall x \in \mathbb{R}, \left \lceil x-1 \right \rceil = \left \lceil x \right \rceil - 1$ as needed. $\quad \quad \quad \quad \quad \blacksquare $

		\item We want to disprove the statement and prove: $\exists x,y \in \mathbb{R}, \left \lceil xy \right \rceil \neq \left \lceil x \right \rceil \left \lfloor y \right \rfloor$

Proof
    
Let  x = 0.5, y = 2, gives:
$$\left \lceil xy \right \rceil = \left \lceil 0.5 \cdot 2 \right \rceil  =  \left \lceil 1 \right \rceil = 1 \neq 2 = 1 \cdot 2 = \left \lceil 0.5 \right \rceil \left \lfloor 2 \right \rfloor = \left \lceil x \right \rceil \left \lfloor y \right \rfloor \ $$

Therefore, we have $\exists x,y \in \mathbb{R}, \left \lceil xy \right \rceil \neq \left \lceil x \right \rceil \left \lfloor y \right \rfloor$ as needed. $\quad \quad \quad \quad \quad \ \blacksquare $
	\end{enumerate}
\end{enumerate}

\newpage

\section{Q3: Induction}
\begin{enumerate}[label=\alph*)]
    \item Proof:
We want to prove $\forall n \in \mathbb{N}, 9\ |\ 11^n - 2^n$.

\textbf{Base Case:} Let n = 0. We want to prove $ 9\ |\ 11^0 - 2^0$, meaning $\exists k_0 \in \mathbb{Z}, 11^0 - 2^0 = 9k_0$:
    
$$11^0 - 2^0 = 1 - 1 = 0 = 9 \cdot 0$$
$\text{when } k_0 = 0,\ 11^0 - 2^0 = 9k_0$, saying that:
$$\text{Therefore, } \exists k_0 \in \mathbb{Z}, \ \text{when n = 0, } 9\ |\ 11^0 - 2^0$$
Hence, we've proven the base case.
    
\textbf{Induction Hypothesis:} Let\ $n \in \mathbb{N}$. Assume $9\ |\ 11^n - 2^n$ which, gives, 
$$\exists k \in \mathbb{Z}, 11^n - 2^n = 9k$$

\textbf{Induction Step:} Want to prove $9\ |\ 11^{n+1} - 2^{n+1}$, meaning $\exists k_1 \in \mathbb{Z}, 11^{n+1} - 2^{n+1} = 9k_1$

Let $w = 11^n + 2k$
\begin{align*}
    &11^{n+1} - 2^{n+1} \\
    =\ &11(11^n) - 2(2^n) \\
    =\ &9(11^n) + 2(11^n) - 2(2^n) \\
    =\ &9(11^n) + 2(11^n - 2^n) \\
    =\ &9(11^n) + 2(9k)  \text{  By Induction Hypothesis}\\
    =\ &9 \cdot (11^n + 2k) \\
    =\ &9 \cdot w \ (9\ | \ 9\cdot w \ is \ True)
\end{align*}
$\text{when } k_1 = w,\ 11^{n+1} - 2^{n+1} = 9k_1$.

Thus, $\exists k_1 \in \mathbb{Z},\ 9\ |\ 11^{n+1} - 2^{n+1} \quad \quad \quad \quad \quad \quad \quad \quad \quad \quad \quad \quad \quad \quad \quad \quad \quad \quad \quad \quad \quad \quad \quad \quad \blacksquare $
\end{enumerate}

\newpage

\begin{enumerate}[label=\alph*)]
	\setcounter{enumi}{1}
	\item Proof: We want to prove $\forall n \in \mathbb{N}, P_{n} =  \prod_{i=0}^{n-1} P_i + 2$

\textbf{Base Case:} Let n = 0. We want to prove $P_0 =  \prod_{i=0}^{0-1}{P_i + 2}$  where $P_n$ is a Pierre Number.

By Pierre Number Definition, 
$$P_0 = 2^{2^0} + 1 = 2 + 1 = 3$$

Since, when $n  < j$, then $\prod_{i=j}^{n} f(i) = 1$, gives, 
$$ \prod_{i=0}^{0-1} P_i + 2 = \prod_{i=0}^{-1} P_i + 2 = 1 + 2 = 3$$

Therefore, when n = 0, $P_0 =  \prod_{i=0}^{0-1}{P_i + 2}$.

Hence, we've proven the base case.
    
\textbf{Induction Hypothesis:} Let $n \in \mathbb{N}$. Assume $P_n =  \prod_{i=0}^{n-1} P_i + 2$

\textbf{Induction Step:} Want to prove $P_{n+1} =  \prod_{i=0}^{(n+1)-1} P_i + 2$
\begin{align*}
    &\prod_{i=0}^{(n+1)-1} P_i + 2 \\
    = &\prod_{i=0}^{n} P_i + 2 \\
    = &\prod_{i=0}^{n-1} P_i \cdot P_n + 2 \\
    = &(P_n - 2) \cdot P_n + 2  \\
    = &{P_n}^2 - 2P_n + 2\text{ (By the I.H.)} \\
    = &(P_n - 1)^2 + 1 \\
    = &(2^{2^n} + 1 - 1)(2^{2^n} + 1 - 1) + 1 \text{ (By definition of } P_n = 2^{2^n} + 1)\\
    = &2^{2(2^n)} + 1 \\
    = &2^{2^{n+1}} + 1 \\ 
\end{align*}
Since, by Pierre Number Definition, $P_{n+1} = 2^{2^{n+1}} + 1$, gives,
$$P_{n+1} =  \prod_{i=0}^{(n+1)-1} P_i + 2$$
Hence, we've proven the induction successfully. 
$$Thus,\ \forall n \in \mathbb{N},\ P_{n} =  \prod_{i=0}^{n-1} P_i + 2$$
$\quad \quad \quad \quad \quad \quad \quad \quad \quad \quad \quad \quad \quad \quad \quad \quad \quad \quad \quad \quad \quad \quad \quad \quad \quad \quad \quad \quad \quad \quad \quad \quad \quad \quad \quad \quad \quad \quad \blacksquare $

\end{enumerate}



\end{document}
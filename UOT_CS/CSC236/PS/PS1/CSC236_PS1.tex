\documentclass[12pt]{article}

\usepackage{enumerate}
\usepackage{graphicx}
\usepackage{geometry}
\usepackage{float}
\usepackage{appendix}
\usepackage{color}
\usepackage{amsmath}
\usepackage{pifont}
\usepackage{fancyhdr}
\usepackage{enumitem}
\usepackage{amssymb}
\fancyhf{}


\geometry{a4paper}
\geometry{left = 2cm}
\geometry{right = 2cm}
\geometry{top = 3cm}
\geometry{bottom = 3cm}

\pagestyle{fancy}
\lhead{CSC236 Fall 2023}
\rhead{Problem Set 1}
\cfoot{\thepage}

\title{CSC236 Problem Set 1}

\author{Xuanqi Wei}

\date{\today}

\begin{document}


\maketitle
\thispagestyle{empty}

\newpage

\tableofcontents
\thispagestyle{empty}

\newpage

\setcounter{page}{1}

\section{Question 1}
\begin{enumerate}[label=(\alph*)]
    \item According to the definition of P:
    $$\forall g_1 \in G_1,\ \exists t_1 \in T_1,\ t_1\ tiles\ g_1 \implies \forall g_2 \in G_2,\ \exists t_2 \in T_2,\ t_2\ tiles\ g_2 $$

    \item 
    Firstly, assume $$\forall g_1 \in G_1,\ \exists t_1 \in T_1,\ t_1\ tiles\ g\text{, which is the antecedent.}$$
    
    Secondly, I will do the consequent part, which: $$Let\ g_2\ be\ an\ arbitrary\ element\ from\ G_2$$ 
    Then, I want to prove that $$\exists t_2 \in T_2,\ t_2\ tiles\ g_2$$ \quad \quad \quad \quad \quad \quad \quad by selecting a satisfying element $t_2$ from $T_2$ 
    and prove the element $t_2$ satisfies $t_2\ tiles\ g_2$.

    \item The diagram above illustrates one instance of $G_2$ grids, which being tiled by triominoes.
    
    Firstly, we already know that for P(1), the statement $\forall g_1 \in G_1,\ \exists t_1 \in T_1,\ t_1\ tiles\ g$ is true which is the antecedent of this direct proof.

    Secondly, the above diagram is an element of the set of all $2^2 \times 2^2$ grid with one square removed, which is an element of $G_2$. 
    By visulising those colorful triominoes, we see a combination triominoes, $t_2$, which is an element of the set of all tilings of elements of $G_2$ using triominoes, belonging to $T_2$, exists and tiles $g_2$.

    Therefore, the diagram above illustrates an instance of that direct proof.

    \item Given the statement to prove: $\forall n \in \mathbb{N},\ P(n)$, which for each natural n you can tile any $2^n \times 2^n$ grid with one cell missing using only triominoes.
    
    \textbf{Proof:} We prove this by Simple Induction on n.


    \textbf{Base Case:} Let $n = 1$.

    Since $G_1$ is the set of all $2^1 \times 2^1$ grids with one cell removed, which by definition is a single triominoe.
    
    Therefore, $\forall g_1 \in G_1,\ \exists t_1 \in T_1,\ t_1\ tiles\ g_1$ is true, which P(1) is true.

    \textbf{Induction Step:} Let $k \in \mathbb{N}$.
    
    \textbf{Induction Hypothesis:} Assume that $P(k)$ is true.

    By Induction Hypothesis, we know that $P(k)$ is true, which $\forall g_k \in G_k,\ \exists t_k \in T_k,\ t_k\ tiles\ g_k $ is true.
    I will take 3 different $g_k$s, the first with right button corner square missing, the second with right top corner square missing, and the third with left top corner square missing.
    I will make the missing corners in these 3 $g_k$s face inwards and add a triomino which will result in getting a `L' shape.
    The remaining $\frac{1}{4}$ place is missing a cell to form a $g_{k+1}$, which can actually be an arbitraty element from $G_k$.
    By Induction Hypothesis, since $\forall g_k \in G_k,\ \exists t_k \in T_k,\ t_k\ tiles\ g_k $ is true, the remaining $G_k$ place can be covered by trimonoes, proving the $P(k+1)$ is true.

    Therefore, we've proved $\forall n \in \mathbb{N},\ P(n)$ is true. 

    $\quad \quad \quad \quad \quad \quad \quad \quad \quad \quad \quad \quad \quad \quad \quad \quad \quad \quad \quad \quad \quad \quad \quad \quad \quad \quad \quad \quad \quad \quad \quad \quad \quad \quad \quad \quad \quad \blacksquare $
\end{enumerate}

\newpage

\section*{Question 2}

\end{document}
\documentclass[fontsize=12pt]{scrartcl}
\usepackage[utf8]{inputenc}
\usepackage{latexsym, amsfonts, fullpage, lscape, cancel, array, lastpage}
\usepackage{mathrsfs, graphicx, amssymb, amsmath, amscd, amsthm, MnSymbol}
\usepackage{titlesec}
\usepackage{enumerate}
\usepackage{graphicx}
\usepackage{geometry}
\usepackage{float}
\usepackage{appendix}
\usepackage{color}
\usepackage{amsmath}
\usepackage{pifont}
\usepackage{fancyhdr}
\usepackage{enumitem}
\usepackage{amssymb}

\titleformat{\section}
  {\normalfont\Large\bfseries}{\S\thesection}{1em}{}

\usepackage[english]{babel}
\usepackage{eucal}

\title{Unit 6(b) Lecture Notes for MAT224}
\author{Xuanqi Wei 1009353209}

\date{28-30 March 2023}

\begin{document}

\maketitle

\noindent
The last topic covered in this course was the Jordan canonical from of a transformation. \textbf{Explain in a few sentences} how each of the following topics we important in defining and being able to find the Jordan canonical form.\\
\\
\begin{itemize}
    \item[(a)] triangularizability: Triangularizability is a concept in linear algebra that refers to the ability to transform a matrix into a triangular form through similarity transformations. A matrix is said to be triangularizable if it is similar to a triangular matrix, which can make certain computations, such as finding eigenvalues, easier.
    \item[(b)] Generalized eigenspaces: Generalized eigenspaces are a concept in linear algebra that relate to the eigenvectors and eigenvalues of a matrix. Specifically, the generalized eigenspace associated with an eigenvalue is the set of all vectors that satisfy a certain equation involving that eigenvalue and the matrix. 
    \item[(c)] $T$-invariant subspaces: In linear algebra, a subspace of a vector space is said to be $T$-invariant if it remains unchanged under the action of a linear operator $T$. In other words, if a vector $v$ belongs to a $T$-invariant subspace, then its image under $T$ also belongs to the same subspace. 
    \item[(d)] nilpotent matrices and transformations: A matrix or a linear transformation is said to be nilpotent if there exists a positive integer $k$ such that the $k$th power of the matrix or the transformation is the zero matrix or the zero transformation, respectively. In other words, repeatedly applying the matrix or the transformation to a vector eventually results in the zero vector.
    \item[(e)] complex vector spaces: A complex vector space is a vector space over the field of complex numbers. In other words, it is a collection of vectors that can be added together and multiplied by complex numbers, satisfying certain axioms such as distributivity and associativity.
    \item[(f)] diagonalizability: Diagonalizability is a concept in linear algebra that refers to the ability to transform a matrix into a diagonal form through similarity transformations. A matrix is said to be diagonalizable if it is similar to a diagonal matrix, which means that it has nonzero entries only on its main diagonal.
    \item[(g)] Spectral Theorem: The Spectral Theorem is a fundamental result in linear algebra that provides a powerful tool for understanding the properties of Hermitian and normal matrices, which are matrices that satisfy certain symmetry and orthogonality conditions, respectively.

The theorem states that every Hermitian matrix is diagonalizable by a unitary matrix, meaning that it can be transformed into a diagonal matrix with real entries by a suitable change of basis. Moreover, the eigenvalues of the Hermitian matrix are real, and the eigenvectors corresponding to distinct eigenvalues are orthogonal.
    \item[(h)] eigenvalues and eigenvectors: Eigenvalues and eigenvectors are fundamental concepts in linear algebra that play a central role in understanding the properties of matrices and linear transformations.

An eigenvector of a matrix or linear transformation is a nonzero vector that, when multiplied by the matrix or transformation, yields a scalar multiple of itself. The corresponding scalar is called the eigenvalue of the matrix or transformation with respect to that eigenvector.
    \item[(i)] characteristic polynomials: In linear algebra, the characteristic polynomial of a square matrix is a polynomial whose roots are the eigenvalues of the matrix. It is defined as the determinant of the matrix minus the identity matrix times a scalar lambda, where lambda is a variable.

The characteristic polynomial of a matrix is a fundamental concept in linear algebra, as it provides a way to compute the eigenvalues of a matrix and to understand its properties. For example, the degree of the characteristic polynomial is equal to the dimension of the matrix, and the coefficients of the polynomial are related to the trace and determinant of the matrix.
    \item[(j)] determinants: In linear algebra, the determinant of a square matrix is a scalar value that encodes important information about the matrix. The determinant is defined as a sum of products of entries of the matrix, where the sign of each term is determined by the arrangement of indices in the product.

The determinant of a matrix can be used to determine whether the matrix is invertible, or equivalently, whether its columns or rows are linearly independent. In particular, a matrix has nonzero determinant if and only if it is invertible, and its determinant gives the scale factor by which the matrix scales volumes or areas.
    \item[(k)] change of basis: Change of basis is a concept in linear algebra that refers to the process of expressing the coordinates of a vector or the entries of a matrix with respect to a different set of basis vectors. A basis is a set of linearly independent vectors that span a vector space, and any vector or matrix can be expressed as a linear combination of its basis vectors.

To change the basis of a vector or matrix, we first express the new basis vectors as linear combinations of the old basis vectors, and then use these expressions to transform the coordinates or entries accordingly. This process involves computing the inverse of a matrix, which can be done using various techniques such as Gaussian elimination or LU decomposition.
    \item[(l)] finding the matrix representation of a linear transformation with respect to a basis:To find the matrix representation of a linear transformation with respect to a basis, we follow these steps:

Choose a basis for the domain and codomain of the linear transformation. Let's call these bases B and C, respectively.
Write the linear transformation as a matrix transformation with respect to the standard basis. Let's call this matrix A.
Find the matrix P that transforms the basis B into the standard basis. This is the matrix whose columns are the basis vectors of B expressed in terms of the standard basis.
Find the matrix Q that transforms the standard basis into the basis C. This is the matrix whose columns are the basis vectors of C expressed in terms of the standard basis.
The matrix representation of the linear transformation with respect to the basis B and C is given by the matrix $P^-1AQ$, where $P^-1$ is the inverse of the matrix P.
    \item[(m)] kernel and image of a transformation, dimension theorem: In linear algebra, the kernel and image of a linear transformation are two fundamental concepts that describe the behavior of the transformation on its domain and codomain.

The kernel of a linear transformation T from a vector space V to a vector space W is the set of all vectors in V that are mapped to the zero vector in W. Symbolically, the kernel of T is denoted by ker(T) or null(T), and is given by:

ker(T) = {v in V : T(v) = 0}

The kernel of a linear transformation is a subspace of the domain, and provides information about the nullity of the transformation.

The image of a linear transformation T from a vector space V to a vector space W is the set of all vectors in W that are the result of applying T to some vector in V. Symbolically, the image of T is denoted by im(T) or range(T), and is given by:

im(T) = {w in W : w = T(v) for some v in V}

The image of a linear transformation is a subspace of the codomain, and provides information about the rank of the transformation.

The dimension theorem, also known as the rank-nullity theorem, relates the dimensions of the kernel and image of a linear transformation, and states that:

dim(ker(T)) + dim(im(T)) = dim(V)

In other words, the sum of the dimensions of the kernel and image of a linear transformation is equal to the dimension of its domain. This theorem has important applications in linear algebra, including the computation of bases for the kernel and image of a transformation, and the determination of when a transformation is invertible.
    \item[(n)] defintion of linear: In mathematics, a function or transformation is said to be linear if it satisfies two properties: additivity and homogeneity.

Additivity means that the function preserves addition, which can be written as:

f(x + y) = f(x) + f(y)

for all vectors x and y in the domain of the function.

Homogeneity means that the function preserves scalar multiplication, which can be written as:

f(kx) = kf(x)

for all vectors x in the domain of the function and all scalars k.

Together, these two properties imply that a linear function is a linear combination of its inputs. In other words, if we apply a linear function to a weighted sum of vectors, the result is equal to the same weighted sum of the function applied to each individual vector.
    \item[(o)] definition of basis: n linear algebra, a basis of a vector space is a set of linearly independent vectors that span the space, which means that any vector in the space can be expressed as a unique linear combination of the basis vectors.

More formally, let V be a vector space over a field F. A set of vectors B = {v1, v2, ..., vn} in V is a basis of V if:

The set B is linearly independent, meaning that for any scalars a1, a2, ..., an in F, the equation a1v1 + a2v2 + ... + anvn = 0 implies that a1 = a2 = ... = an = 0.
The set B spans V, meaning that for any vector v in V, there exist scalars a1, a2, ..., an in F such that v = a1v1 + a2v2 + ... + anvn.

    \item[(p)] linear independence and dependence: In linear algebra, a set of vectors is said to be linearly independent if none of the vectors in the set can be expressed as a linear combination of the others.

More formally, let V be a vector space over a field F, and let {v1, v2, ..., vn} be a set of vectors in V. Then, the set is said to be linearly independent if the only solution to the equation:

a1v1 + a2v2 + ... + anvn = 0

is the trivial solution, where a1, a2, ..., an are scalars in F and 0 denotes the zero vector in V.

If a set of vectors is not linearly independent, it is said to be linearly dependent. This means that at least one of the vectors in the set can be expressed as a linear combination of the others.
    \item[(q)] definition of ``the span of a set'': In linear algebra, the span of a set of vectors is the set of all possible linear combinations of those vectors.

More formally, let V be a vector space over a field F, and let S = {v1, v2, ..., vn} be a set of vectors in V. Then, the span of S, denoted span(S), is defined as the set of all linear combinations of the vectors in S, i.e.,

span(S) = {a1v1 + a2v2 + ... + anvn : a1, a2, ..., an are scalars in F}.
    \item[(r)] dimension of vector spaces and their subspaces: In linear algebra, the dimension of a vector space is the number of vectors in any basis for that space.

More formally, let V be a vector space over a field F. A basis for V is a linearly independent set of vectors in V that spans the space, i.e., every vector in V can be expressed uniquely as a linear combination of the vectors in the basis. The dimension of V, denoted dim(V), is defined as the number of vectors in any basis for V.

The dimension of a vector space has many important consequences in linear algebra. For example, it determines the number of coordinates needed to specify a vector in that space, and it provides a bound on the size of any linearly independent set of vectors in the space. It also plays a key role in the study of linear transformations, as it affects the rank and nullity of these objects.

If U is a subspace of V, then the dimension of U, denoted dim(U), is defined as the number of vectors in any basis for U. Since any linearly independent set of vectors in U can be extended to a basis of V, it follows that dim(U) ≤ dim(V). Moreover, the dimension of the intersection of two subspaces U and W satisfies the inequality

$$dim(U ∩ W) \leq min{dim(U), dim(W)}$$

In particular, if U and W are distinct subspaces of V such that dim(U) + dim(W) > dim(V), then U and W cannot intersect trivially, i.e., they must have a nontrivial intersection. This is known as the dimension theorem or the rank-nullity theorem in linear algebra.
\end{itemize}


\end{document}
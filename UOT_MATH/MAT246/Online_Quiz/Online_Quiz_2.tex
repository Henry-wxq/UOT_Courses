\documentclass[12pt]{article}

\usepackage{enumerate}
\usepackage{graphicx}
\usepackage{geometry}
\usepackage{float}
\usepackage{appendix}
\usepackage{color}
\usepackage{amsmath}
\usepackage{pifont}
\usepackage{fancyhdr}
\usepackage{enumitem}
\usepackage{amssymb}
\fancyhf{}


\geometry{a4paper}
\geometry{left = 2cm}
\geometry{right = 2cm}
\geometry{top = 3cm}
\geometry{bottom = 3cm}

\pagestyle{fancy}
\lhead{MAT246 Fall 2023-2024}
\rhead{Online Quiz 2}
\cfoot{\thepage}

\title{MAT246 Online Quiz 2}

\author{Xuanqi Wei 1009353209}

\date{\today}

\begin{document}


\maketitle
\thispagestyle{empty}

\newpage

\setcounter{page}{1}

\section{Question 1}

\textbf{Proof:} Let $k \in \mathbb{N}$. Let $n \in \mathbb{N}$

\noindent Assume k does not divide $n^2$, which $n^2$ has no factorization, which $n^2 \neq k \cdot p$, where $p \in \mathbb{N}$.
Meaning k is not a divisor of $n^2$.(Assumption 1)

\noindent \textbf{Assume for contradiction:} k can divide n, which n has a factorization $n = k \cdot q$, where $q \in \mathbb{N}$.

\noindent Since $n^2 = n \cdot n$ (from the definition of $n^2$), gives
\begin{align*}
    n^2 &= n \cdot n\ \text{(According to the definition of $n^2$)} \\
    &= (k \cdot q) \cdot (k \cdot q) \ \text{(According to the Assumption for Contradiction)} \\
    &= k \cdot (q \cdot k \cdot q) \ \text{(According to the Commutative Law of Multiplication)}
\end{align*}

\noindent Since $q \in \mathbb{N}$, $k \in \mathbb{N}$, we have $(q \cdot k \cdot q) \in \mathbb{N}$.
Thus, $\exists p \in \mathbb{N},\ s.t.\ p=(q \cdot k \cdot q)$ where $n^2 = k \cdot p $, which is a factorization of $n^2$, contrdicts to the Assumption 1 that $n^2$ has no factorization.

\noindent Therefore, to conclude, for natural numbers k and n, if k does not divide $n^2$, then k cannot divide n either.

$\quad \quad \quad \quad \quad \quad \quad \quad \quad \quad \quad \quad \quad \quad \quad \quad \quad \quad \quad \quad \quad \quad \quad \quad \quad \quad \quad \quad \quad \quad \quad \quad \quad \quad \quad \quad \quad \quad  \blacksquare $

\section{Question 2}

\textbf{Proof:} Let $k \in \mathbb{N}.$

\noindent Assume there are exactly k many natural numbers r, such that $1 \leq r \leq k$.

\noindent To prove there is exactly (k+1) many natural numbers r such that $1 \leq r \leq (k+1)$, there contains mainly parts, there are no less than $(k+1)$ many natural numbers r and there are no more than $(k+1)$ many natural numbers r.

\noindent \textbf{Part 1:} There are no less than $(k+1)$ many natural numbers r such that $1 \leq r \leq (k+1)$.

\noindent Considering the set of natural numbers between 1 and k, which $1 \leq r \leq k$, \{1, 2, 3, ..., k\}, which contains exactly k natural numbers.
Extending the set listed above to include $(k+1)$, the set becoms \{1, 2, 3, ..., k, (k+1) \}. This set contains k natural numbers from the original set \{1, 2, 3, ..., k\} and one more element, which is $(k+1)$.
Since the original set contains k natural numbers, and I've added one more $(k+1)$ to form the new set \{1, 2, 3, ..., k, (k+1) \}, there are no fewer than $(k+1)$ natural numbers r, which $1 \leq r \leq (k+1)$.

\noindent \textbf{Part 2:} There are no more than $(k+1)$ many natural numbers r such that $1 \leq r \leq (k+1)$.

\noindent \textbf{Assume for contradiction:} There are more than $(k+1)$ many natural numbers r such that $1 \leq r \leq (k+1)$.

\noindent Saying there are p natural numbers between 1 and $(k+1)$, according to the assuption for contradiction, $p > (k+1)$. 
Representing the the set of natural numbers between 1 and $(k+1)$ by using the index, gives: \{$x_1,\ x_2,\ x_3,\ ...,\ x_p$\}, such that $1 \leq x_1 < x_2 < ... < x_p \leq (k+1)$.

\noindent Since all the numbers $x_i$ in the set are distinct, there are p numbers in the set. However, since $p > (k+1)$, this means that at least one of the number $r=x_i$ in the set must be greater than (k+1), which contradicts to the assumption for contradiction as we assume all the numbers in \{$x_1,\ x_2,\ x_3,\ ...,\ x_p$\} are between 1 and $(k+1)$.

\noindent Combining both proofs, we have shown that there are there are no less than $(k+1)$ many natural numbers r and there are no more than $(k+1)$ many natural numbers r such that $1 \leq r \leq (k+1)$, which there is exactly (k+1) many natural numbers r such that $1 \leq r \leq (k+1)$.

$\quad \quad \quad \quad \quad \quad \quad \quad \quad \quad \quad \quad \quad \quad \quad \quad \quad \quad \quad \quad \quad \quad \quad \quad \quad \quad \quad \quad \quad \quad \quad \quad \quad \quad \quad \quad \quad \quad  \blacksquare $

\newpage

\section{Question 3}
\textbf{Proof:} Let S be the set of all natural numbers for which the theorem, $\forall n \in \mathbb{N}$, there are exactly n many natural numbers r such that $1 \leq r \leq n$, is true.
We want to show that S contains all of the natural numbers.
We do this by showing that S has properties A and B.

\noindent For property A, the base case, we need to check that there is exactly one natural number r such that $1 \leq r \leq 1$.
It's apparent that, in this case, $r=1$, which there is exactly one natural number satisfying.

\noindent To verify property B, let k be in S. We must show that $(k+1)$ is in S.
For a natural number k, assume there are exactly k many natural numbers r, such that $1 \leq r \leq k$, which is the Induction Hypothesis.
Show that there is exactly $(k+1)$ many natural numbers r such that $1 \leq r \leq (k+1)$, which is the Induction Conclusion.

\noindent We observed that for property B, we've already proved it in Question 2, which it's true.

\noindent Therefore, S is the set of natural numbers by the Principle of Mathematical Induction. To conclude, for each natural number n, there are exactly n many natural numbers r such that $1 \leq r \leq n$.

$\quad \quad \quad \quad \quad \quad \quad \quad \quad \quad \quad \quad \quad \quad \quad \quad \quad \quad \quad \quad \quad \quad \quad \quad \quad \quad \quad \quad \quad \quad \quad \quad \quad \quad \quad \quad \quad \quad  \blacksquare $

\end{document}